\begingroup
\selectlanguage{english}
\begin{abstract}
A distanza di anni dalla sua nascita la Blockchain rappresenta un paradigma e una piattaforma di innovazione che permette di dare nuove risposte a tanti e diversi bisogni di imprese, organizzazioni, cittadini e consumatori. 
Tra questi spicca il settore della finanza decentralizzata (DeFi), con il quale è possibile implementare piattaforme che consentono forme di scambio del valore, sotto forma di token, che non richiedono garanti o intermediari.
Il presente studio mira alla realizzazione di applicazioni Blockchain per la finanza decentralizzata, tramite l'utilizzo di tecnologie moderne come React e Solidity.
Tali implementazioni hanno lo scopo di dimostrare che le applicazioni DeFi possono essere di pubblico utilizzo, senza la necessità di grandi conoscenze preliminari. 

Le applicazioni decentralizzate (Dapp) sviluppate nel seguente lavoro di tesi sono due: la prima consiste in una piattaforma di staking di criptovalute (solo la parte front-end in React), la seconda, invece, riguarda il mondo degli exchange decentralizzati, per la precisione, degli Automatic Market Maker (AMM); la logica di quest'ultima è basata sugli smart-contract scritti in Solidity con un front-end sviluppato in React.


\end{abstract}
\endgroup
