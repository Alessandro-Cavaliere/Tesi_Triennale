
\phantomsection\addcontentsline{toc}{chapter}{Introduzione}
\chapter*{Introduzione}
\lhead{\textit{Introduzione}}
Quando si parla di valute digitali, la prima \textit{key word} che sorge alla mente è “Bitcoin”.
Bitcoin portò alla ribalta mondiale la tecnologia Blockchain con il suoi clamorosi aumenti di valore nel corso della storia, innescando la rivoluzione degli scambi \textit{peer-to-peer} monetari.
Questi scambi monetari rappresentano l'ondata evolutiva che, negli ultimi anni, la Blockchain sta attraversando.
Un'altra svolta che caratterizza il mondo della Blockchain è stata quella dell'introduzione degli smart-contract su Ethereum con il conseguente scoppio delle applicazioni basate su Blockchain, le cosiddette applicazioni decentralizzate. 

L'internet, così come lo conosciamo oggi, ossia come puro mezzo di diffusione d' informazioni (Intenet of Information), si sta evolvendo in un mezzo di diffusione di valore (Intenet of Value): DeFi o finanza decentralizzata. L'obiettivo di questo elaborato è quello di fornire una panoramica dei possibili sviluppi della finanza decentralizzata con concreti esempi di implementazione di decentralized App che concernano tutte le logiche intrinseche della DeFi. 

L'elaborato è organizzato come di seguito:  nel capitolo \ref{cap1} verrà introdotta la tecnologia blockchain, spiegato il funzionamento, la sua storia, la sua evoluzione nel tempo ed eventuali possibili sviluppi futuri. 

Nel capitolo \ref{cap2} verranno descritti gli strumenti di sviluppo utilizzati andando dapprima a presentare come vengono organizzati gli applicativi in termini di sviluppo per poi passare all'introduzione delle tecnologie di back-end e front-end utilizzate.

In conclusione, il capitolo \ref{cap3} consiste nell'esposizione degli applicativi sviluppati, facendo, prima, un piccolo excursus della logica su cui si basano, per poi passare al puro aspetto implementativo, mostrando i codici che danno vita alle varie interfacce e rendono le operazioni possibili.